\chapter{Methodology}


\section{Description of methods to solve the problem}

I plan to use AI or neural network to analyze requirement documents, and eventually transform them to executable programs.
At the first step, I will transform to a blockchain smart contract.
I will have to analyze the characteristics of smart contract, including parallel properties and data persistence, and find out what programs can be safely transformed to smart contract.

First I will try rule-based transformation where the rules are hand-coded. Second, I will make use of neural networks to do the transformation. In addition, I will publish papers on the theory part of smart contracts as they are like distributed programs.

I first will use CoCoME as the study case, transform it to a blockchain smart contract. Then I will evaluate my algorithm on other study cases, e.g., SLEX-Web.

After my conversion tool is implemented, I hope to let people use it so that I can find out how much my tool can actually save people's time, and see to what extend people have to revise the auto-generated code. Hence, I can list these numbers in my paper to prove the effectiveness of my tool.

When a prototype is generated by my tool, it is still far from being a product, as the state-of-the-art code generation systems leave holes where the system is uncertain or the operation involves  third-party API or sorting. In addition, the prototype must be refactored and improved to meet non-functional requirements, e.g., aesthetic guidelines.

In the further coding, developers make commits which improves the system bit by bit, and each commit is essentially a diff. To process diff, neural network is again a key. I expect to use text vectorization layers to split text into words, and use word embeddings to convert each word into a vector of decimals.
The remaining of the neural network can take the simplest dense layers, or more advanced convolutional neural network~\cite{albawi2017understanding}, recurrent neural network~\cite{tarwani2017survey}, long short-term memory~\cite{skovajsova2017long}, etc.

A similar evaluation is desired to measure the effectiveness of the commit assistance tool. I may divide participants into two groups, one group uses the tool and the other does not use. I will compare how much productivity increase the group using the tool has comparing to the other group.


