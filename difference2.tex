\documentclass[preview,svgnames, border={0.5cm}]{standalone}
\usepackage{tikz}


\usepackage{listings}
\lstset{
	%  %行号
	%  %numbers=left,
	%  %背景框
	%  %framexleftmargin=10mm,
	%  %frame=none,
	%  %背景色
	%  %backgroundcolor=\color[rgb]{1,1,0.76},
	%  %样式
	keywordstyle=\color{blue}\bfseries,
	%  identifierstyle=\bf,
	%  numberstyle=\color[RGB]{0,192,192}, %行号的样式
	commentstyle=\it\color[RGB]{0,96,96},
	stringstyle=\rmfamily\slshape\color[RGB]{128,0,0},
	%  %显示空格
	%  %showstringspaces=false,
	basicstyle=\ttfamily\scriptsize,
	columns=flexible, %修正大写字母间距过小
	breaklines=true, %对过长的代码自动换行
	%  escapechar=!,
	%  morekeywords={BEGIN}
}

\usepackage{caption}
\usepackage{subcaption}

\usetikzlibrary{arrows,shapes,positioning,calc, shapes.geometric,shapes.arrows}


\begin{document}
\tikzset{Arrow/.style={thick, ->, >=stealth}}
\tikzset{Target/.style={baseline, inner sep=0pt, remember picture}}
\tikzstyle{Text} = [/utils/exec={\sffamily}, text centered, align=center]

\begin{figure}
\centering
\begin{subfigure}[T]{0.4\textwidth}
\begin{lstlisting}[language=java,basicstyle=\tiny, escapechar=\%,]
public class AddNumber {
  %\tikz[style=Target, anchor=base]{\node (private) {\color{blue}{\bfseries private}}}% int sum = 0;

  public void add(int n) {
    sum += n;
  }

  public int getSum() {
    return sum;
  }
}
\end{lstlisting}
\end{subfigure}
\hfill
\begin{subfigure}[T]{0.55\textwidth}
\begin{lstlisting}[language=java,basicstyle=\tiny, escapechar=\%,]
import org.hyperledger.fabric.*;

@Contract
public class AddNumber implements ContractInterface {
  private final Genson genson = new Genson();

  @Transaction(intent = Transaction.TYPE.%\tikz[style=Target, anchor=base]{\node (submit) {SUBMIT}}%)
  public void add(final Context ctx, int n) {
    ChaincodeStub stub = ctx.getStub();

    %\tikz[style=Target, anchor=base]{\node (sl1) {}}%int sum = Integer.parseInt(stub.getStringState("sum"));
    sum += n;

    %\tikz[style=Target, anchor=base]{\node (sl2) {}}%stub.putStringState("sum", String.valueOf(sum));
  }

  @Transaction(intent = Transaction.TYPE.%\tikz[style=Target, anchor=base]{\node (evaluate) {EVALUATE}}%)
  public int getSum(final Context ctx) {
    ChaincodeStub stub = ctx.getStub();
    %\tikz[style=Target, anchor=base]{\node (sl3) {}}%return Integer.parseInt(stub.getStringState("sum"));
  }
}
\end{lstlisting}
\end{subfigure}
\end{figure}

\begin{tikzpicture}[remember picture,overlay]
\node[above=0.5cm of private, xshift=1.5cm] (private-ex) {no private data in blockchain};
\draw[Arrow] (private) -- (private-ex);

\node[Text, left=1cm of sl3] (sl-ex) {Internal states must be\\ loaded/saved to the chain.};
\draw[Arrow] (sl1) -- (sl-ex);
\draw[Arrow] (sl2) -- (sl-ex);
\draw[Arrow] (sl3) -- (sl-ex);

\node[Text, above=1.5cm of submit] (submit-ex) {a set function,\\ adds a new block};
\draw[Arrow] (submit) -- (submit-ex);

\node[Text, below=1cm of evaluate] (evaluate-ex) {a get function,\\ no new block};
\draw[Arrow] (evaluate) -- (evaluate-ex);
\end{tikzpicture}


\end{document}