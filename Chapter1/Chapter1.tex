\chapter{Introduction}



\section{General overview of the field of study}
Software development is the process of conceiving, specifying, designing, programming, documenting, testing, and bug fixing involved in creating and maintaining applications, frameworks, or other software components.
%Software development involves writing and maintaining the source code, but in a broader sense, it includes all processes from the conception of the desired software through to the final manifestation of the software, typically in a planned and structured process.
%Software development also includes research, new development, prototyping, modification, reuse, re-engineering, maintenance, or any other activities that result in software products.

%waterfall
Software development typically involves the following phases, requirements elicitation, analysis, design, coding, testing, and operation. We notice major difficulties in software development happen in the requirements elicitation phase and the coding or implementation phase.

\section{Motivation}

Requirements errors are one of the causes leading failings in software projects~\cite{sutcliffe1999tracing}.
%Careful requirements modeling along with systematic validation helps to reduce the uncertainty about target systems [2], [3]. The goal of requirements validation is to construct the consistent requirements for the needs of target users [4]. However, this process is complicated, and it can be hard to produce a correct and complete requirements specification. The complexity is due to the following interrelated attributes [5]–[7]:
%1) the complexity of application domains and business processes;2) the uncertainty of clients and domain experts about their needs;3) the lack of the understanding of system developers about application domains;4) the difficulties of the understanding between system developers and clients.
Rapid prototyping is an effective approach to requirements validation and evolution via an executable model of a software system to demonstrate concepts, discover requirements errors and find possible fixing solutions, and discover missing requirements~\cite{kordon2002introduction}.
Therefore, it is very desirable to have a tool that generates prototypes directly from requirements automatically.


%todo: neural networks

%todo: blockchain

In the development phase, developers typically use a version control system,
which organizes a group of interrelated changes to the code, sometimes called diff, as a commit, along with a commit message explaining the rational behind the changes or other useful information.
%A version control system tracks and manages changes to software code thus helps the team manage changes to source code over time.
However, commit messages can be incomplete or inaccurate~\cite{buse2010automatically} and undermine developers' ability to understand and find proper commits.
%For example, when writing commit messages,
%a developer may say only identifiers were changed
%if he or she didn't check the diff very carefully,
%while the developer actually changed some white spaces or fixed a few minor bugs.
%In order to enforce software quality and the quality of commit messages,
%code review is introduced as a process in software engineering,
%where another developer reviews the committed code and
%the commit messages made by the former developer~\cite{shimagaki2016study}.

We believe that commits can be classified based on their diff.
Proper classification of commits can remedy incomplete or inaccurate commit messages written manually, and
guide code reviewers to pick commits to review.



