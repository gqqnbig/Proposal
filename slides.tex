\documentclass[xcolor=svgnames]{beamer}
\setbeamertemplate{footline}[frame number]
\setbeamertemplate{navigation symbols}{}

\usepackage{tikz}
\usetikzlibrary{arrows,shapes,positioning,calc,fadings, shapes.geometric}

\usepackage{listings}
\lstset{
	%  %行号
	%  %numbers=left,
	%  %背景框
	%  %framexleftmargin=10mm,
	%  %frame=none,
	%  %背景色
	%  %backgroundcolor=\color[rgb]{1,1,0.76},
	%  %样式
	keywordstyle=\color{blue}\bfseries,
	%  identifierstyle=\bf,
	%  numberstyle=\color[RGB]{0,192,192}, %行号的样式
	commentstyle=\it\color[RGB]{0,96,96},
	stringstyle=\rmfamily\slshape\color[RGB]{128,0,0},
	%  %显示空格
	%  %showstringspaces=false,
	basicstyle=\ttfamily\scriptsize,
	columns=flexible, %修正大写字母间距过小
	breaklines=true, %对过长的代码自动换行
	%  morekeywords={BEGIN}
}

\definecolor{diffstart}{named}{Grey}
\definecolor{diffincl}{rgb}{0, 0.35, 0}
\definecolor{diffrem}{rgb}{0.72, 0, 0}
\lstdefinelanguage{diff}{
	%basicstyle=\ttfamily\small,
	morecomment=[f][\color{diffstart}]{@@},
	morecomment=[f][\color{diffincl}]{+\ },
	morecomment=[f][\color{diffrem}]{-\ },
}


\usepackage[group-separator={,},group-minimum-digits={3}]{siunitx}

\usepackage{multirow}
\usepackage{pgfplots}
\usepackage{contour}
\usepackage{caption}
\usepackage{subcaption}

\newcommand{\code}[1]{\texttt{#1}}

\tikzstyle{every picture}+=[remember picture]

\tikzstyle target=[baseline, inner sep=0pt, remember picture]

\title{Identification of Functional Implementation in Conventional Applications and Transformation to Smart Contracts}
\author{Qiqi Gu\\P0907870}
\institute{%
\tikz[style=target]{\node [anchor=base, baseline, inner sep=0pt, align=center] (mpi text)
	{School of Applied Sciences \\ Macao Polytechnic Institute, Macao SAR, China};
}%
}
\date{March 2, 2022}

\begin{document}

{
\usebackgroundtemplate{\includegraphics[height=\paperheight]{background.jpg}}%

\begin{frame}[plain]
	\maketitle

	\begin{tikzpicture}[overlay]
	\node[inner sep=0pt, left=0.2cm of mpi text]{\includegraphics[width=1.2cm]{mpi-logo.png}};
	\end{tikzpicture}
\end{frame}
}

\begin{frame}{Outline}
	\tableofcontents
\end{frame}

\section{Background}
\begin{frame}{Background}
\begin{itemize}
	\item Modern software development relies on version control systems, e.g., git.
%	\item Each commit consists of a change set and a commit message.
	\item Developers make pull requests and when the PR is reviewed, it gets merged to the master branch.
	\item Reviewing different PR requires varied skills.
	\begin{itemize}
		\item functional, refactoring, docs, tests, etc.
	\end{itemize}
\end{itemize}
\end{frame}

\begin{frame}[t]{Background}
\begin{itemize}
\item 2/3 of open source projects on GitHub have just one or two maintainers~\cite{avelino_what_2015}.
	\begin{itemize}
		\item<only@2-> fastlane has 1000+ contributors but only 7 maintainers.
		\item<only@3-> pandas has 2500+ contributors but 35 maintainers.
	\end{itemize}
\item<only@4-> Reviewing code commits is a burden for the slim maintainers.
\item<only@5-> We believe that commits can be classified based on their diff.
\item<only@5-> Proper classification of commits can remedy incomplete or inaccurate commit messages written manually.
\end{itemize}

\end{frame}


\section{Contribution}

\begin{frame}{Contribution}
\begin{itemize}
	\item Classify a commit (or a pull request) whether the commit only changes an identifier.
	\item Created a neural network classification model with diff files as input, telling whether the input is identifier renaming or not.
	\item The model can take part in Continuous Integration and assign pull requests to proper reviewers.
	\item Trained on Java 7
\end{itemize}
\end{frame}


% lstlisting requires fragile option.
\begin{frame}[fragile]{Diff Format}
Each commit can be viewed as a diff file.

\onslide<3->{Only
\tikz[style=target]{\node [anchor=base, baseline, inner sep=0pt] (a) {line 8 to 15 };
\draw[red, overlay]([shift={(-.1,.1)}]a.north west) rectangle ([shift={(.1,-.1)}]a.south east);
}
are fed into our model.}

\begin{figure}
\begin{lstlisting}[language=diff, breaklines=true, numbers=left, xleftmargin=2em, escapechar=\%,]
diff --git a/src/.../TldPatterns.java b/src/.../TldPatterns.java
old mode 100644
new mode 100755
diff --git a/src/.../Chars.java b/src/.../Chars.java
index a0cf5bd..e26dca8 100644
--- a/src/.../Chars.java
+++ b/src/.../Chars.java
%\tikz[style=target]{\node [anchor=base, yshift=0.2cm] (tl) {}}%@@ -43,7 +43,7 @@ import java.util.RandomAccess; %\tikz[style=target]{\node [anchor=base, yshift=0.2cm] (tr) {}}%
* @author Kevin Bourrillion
* @since 1
*/
- public class Chars {
+ public final class Chars {
    private Chars() {}
%\tikz[style=target]{\node [anchor=base] (bl) {}}%}
\ No newline at end of file
\end{lstlisting}
\end{figure}

\onslide<2->
\begin{tikzpicture}[remember picture,overlay]
	\node[right=of tr, anchor=south, yshift=0.3cm] (ex) {hunk header};
	\path[blue, -, transform canvas={yshift=-0.1cm,xshift=-2pt}] (ex) edge (tr);
\end{tikzpicture}

\onslide<3->
\begin{tikzpicture}[remember picture,overlay]
	\path[red, -] (tl.north west) edge (bl.south west);
	% \path[blue, -] doesn't need draw option because - is an arrow option. The draw is implicit.
	\path[red, draw] (bl.south west) -| (tr.north east);
	\path[red, -] (tr.north east) edge (tl.north west);
\end{tikzpicture}
\end{frame}

\begin{frame}{Overview of the workflow}
The two operations on the left are for creating labeled dataset.

The three on the right are model layers.

\end{frame}

\section{Create Labeled Dataset}
\begin{frame}{Create Labeled Dataset}
There are no off-the-shelf datasets suitable for our purpose.

\begin{figure}
\centering
\tikzstyle{dataset} = [draw=black, minimum width=1cm, align=center]
\tikzstyle{arrow} = [thick, ->, >=stealth]
\begin{tikzpicture}
\node (jiang) [dataset] {... ?\\\color{Green}{... ?}\\\color{red}{... ?}\\... ?};
\node (exp1) [below=0.3cm of jiang, align=center] {Jiang dataset~\cite{jiang2017}\\2 million commits};

\onslide<2->{
\node (clean) [dataset, right=1.8cm of jiang] {... ?\\... ?\\... ?};
\node [align=center, anchor=north] at (clean |- exp1.north) {62K commits};
\draw [arrow] (jiang) -- (clean);
}

\onslide<3->{
\node (our) [dataset, right=1.8cm of clean] {... ?\\... ?\\... ?\\... ?\\... ?};
\node [align=center, anchor=north] at (our |- exp1.north) {Our dataset\\73080 commits};
\draw [arrow] (clean) -- (our);
}
\end{tikzpicture}
\end{figure}
\end{frame}

\begin{frame}[t]{Create Labeled Dataset}
\begin{itemize}
\item  We implemented a diff syntax analyzer in {\sc antlr} 4 with Java 7 grammar, which labeled individual commits as ``yes'' or ``no''.
\item<only@2-> The code in the beginning and at the end of a hunk may not be of complete syntax units.
\item<only@2-> {\sc antlr} 4 comes with error recovery heuristics summarized as single-token insertion and single-token deletion~\cite{parr2013definitive},
so that it can figure out the type of a unit by the longest match.
\end{itemize}

\begin{onlyenv}<1>
\begin{figure}
\centering
\tikzstyle{dataset} = [draw=black, minimum width=1cm, align=center]
\tikzstyle{arrow} = [thick, ->, >=stealth]
\begin{tikzpicture}
\node (our) [dataset] {... ?\\... ?\\... ?\\... ?\\... ?};
\node [align=center, below=0.3cm of our] {Our dataset\\73080 commits};
\node (labeled) [dataset, right=2.5cm of our] {... no\\... yes\\... no\\... no\\... yes};
\node [align=center, below=0.3cm of labeled] {Our labeled dataset};
\draw [arrow] (our) -- (labeled);
\end{tikzpicture}
\end{figure}
\end{onlyenv}

\end{frame}

\begin{frame}[fragile]{Create Labeled Dataset}

\begin{figure}
\begin{lstlisting}[language=diff, breaklines=true, numbers=left, xleftmargin=2em, firstnumber=9, escapechar=\%,]
%\tikz[style=target]{\node [anchor=base, yshift=0.2cm] (c) {}}%* @author Kevin Bourrillion
* @since 1
*/
- public class Chars {
+ public final class Chars {
    private Chars() {}
}
\end{lstlisting}
\end{figure}

\begin{tikzpicture}[remember picture,overlay]
	\node[right=of c, anchor=south west, yshift=0.3cm, xshift=1cm, text centered, align=left] (ex) {Antlr adds /* so that \\ this piece becomes a block comment.};
	\path[blue, draw, ->, transform canvas={yshift=-0.1cm,xshift=-2pt}] (ex) -| (c);
\end{tikzpicture}
\end{frame}

\begin{frame}{Definition of Identifier Renaming}
\begin{enumerate}
	\item Renaming of only one identifier is allowed.
	If a commit changes multiple identifiers, the commit is classified as ``no''.

	\item Renaming of method overloads is treated as a single renaming.
	Renaming of a class leading to renaming of its constructors is a single renaming.

	\item Package renaming is labeled as ``no''.

	\item If comments are changed, the diff is ``no''.

	\item An empty diff file is ``no''.
\end{enumerate}
\end{frame}

\section{Create Neural Network Model}
\begin{frame}{Model}
\tikzstyle{io} = [trapezium, trapezium left angle=70, trapezium right angle=110, draw=black, minimum width=3cm, text centered, align=center]
\tikzstyle{process} = [rectangle, draw=black, inner sep=0.1cm, text centered, align=center]

\tikzstyle{arrow} = [thick, ->, >=stealth]
\begin{figure}
	\begin{tikzpicture}[node distance=0.5cm and 1cm]
		\node (text) [process, anchor=south west] {Text vectorization \\ (standaize, split, n-gram)};
		\node (embedding) [process, below=of text] {Word embedding};
		\node (dense) [process, below=of embedding] {Dense layers};
		\node (output) [io, below=of dense] {Probabilities};

		\draw [arrow] (text) -- (embedding);
		\draw [arrow] (embedding) -- (dense);
		\draw [arrow] (dense) -- (output);
	\end{tikzpicture}
\end{figure}
\end{frame}

\begin{frame}{Model}
\begin{itemize}
\item Text vectorization layer standardizes all numbers to an internal token.
\item Splits the text by white spaces and punctuation characters.
\item We keep multi-word identifiers intact because it achieves higher accuracy~\cite{haiduc2010use}.
\item Tokens are combined into 2-grams.
\item The layer limits maximum vocabulary to 10K and the infrequent ones are discarded.
\item When the text vectorization layer sees a word not in the capped vocabulary, the word will be represented by a special \code{UNK} token.
\end{itemize}
\end{frame}

\begin{frame}[fragile]{Model - Text Vectorization}

\newcommand{\tokenbox}[1]{{\setlength{\fboxsep}{1pt}\fbox{\tt\vphantom{\textbackslash}#1}}}
\begin{onlyenv}<1>
\begin{figure}
\begin{lstlisting}[language=diff, breaklines=true, escapechar=\%,]
- public void runDefaultOutgoing(Execution exceution) {
+ public void runDefaultOutgoing(Execution execution) {
-   performOutgoingBehavior(exceution, true, null);
+   performOutgoingBehavior(execution, true, null);
}
\end{lstlisting}
\end{figure}
\end{onlyenv}

\begin{onlyenv}<2-9>
\begin{figure}
\begin{lstlisting}[language=diff, breaklines=true, escapechar=\%,]
24 %\alt<3->{85}{public void}% %\alt<4->{1}{runDefaultOutgoing}%%\alt<5->{ 4 }{(}%%\alt<6->{8489}{Execution}% %\alt<7->{1}{exceution}%%\alt<8->{ 5}{)}% %\alt<9->{15}{\{}%
+ %\alt<3->{85}{public void}% %\alt<4->{1}{runDefaultOutgoing}%%\alt<5->{ 4 }{(}%%\alt<6->{8489}{Execution}% execution%\alt<8->{ 5}{)}% %\alt<9->{15}{\{}%
24   performOutgoingBehavior%\alt<5->{ 4 }{(}%exceution, true, null%\alt<8->{ 5}{)}%;
+   performOutgoingBehavior%\alt<5->{ 4 }{(}%execution, true, null%\alt<8->{ 5}{)}%;
}
\end{lstlisting}
\end{figure}
\end{onlyenv}

\begin{onlyenv}<10->
\begin{figure}
\begin{lstlisting}[language={}, breaklines=true, escapechar=\%,]
24 85 1 4 8489 1 5 15
23 85 1 4 8489 %\alt<11->{4420}{execution}%%\alt<8->{ 5}{)}% %\alt<9->{15}{\{}%
24 performOutgoingBehavior(%\alt<11->{4420}{execution}%, true, null);
23 performOutgoingBehavior(%\alt<11->{4420}{execution}%, true, null);
}
\end{lstlisting}
\end{figure}
\end{onlyenv}

vocabulary:\\
\tokenbox{{\textbackslash}n -} -> 24\\
\tokenbox{public void} -> 85\\
\tokenbox{runDefaultOutgoing} -> \code{UNK} (1)\\
\tokenbox{(} -> 4\\
\tokenbox{Execution} -> 8489\\
\tokenbox{execution} -> 4220\\
\tokenbox{)} -> 5\\
\tokenbox{\{} -> 15\\
\tokenbox{{\textbackslash}n +} -> 23\\
...\\
\end{frame}

\begin{frame}[fragile]{Model - Text Vectorization}

\begin{tikzpicture}
\node (text) {\begin{lstlisting}[language=diff, breaklines=true]
- public void runDefaultOutgoing(Execution exceution) {
+ public void runDefaultOutgoing(Execution execution) {
-   performOutgoingBehavior(exceution, true, null);
+   performOutgoingBehavior(execution, true, null);
}
\end{lstlisting}};
\node (matrix) [below=of text] {$[ 24, 85, 1, 4, 8489, 1, 5, 15, 23, 85, 1, 4, 8489, 4420, 5, 15, \cdots]$};

\draw [thick, ->, >=stealth] (text) -- (matrix);
\end{tikzpicture}



\end{frame}

\begin{frame}{Model}
\begin{itemize}
\item The word embedding layer transforms each token to a vector of length 16. It was trained from our diff dataset.
\item The input is flattened and goes through the dense layers of size 100, 10, and 1, respectively with RELU as activation function.
\item The total number of parameters of our neural network model is \num{6428332}.
\end{itemize}
\end{frame}

\section{Evaluation}


\begin{frame}{Dataset}
Our dataset has \num{73080} examples but during evaluation we ignored examples bigger than 10 KB, getting a filtered dataset of \num{72079} files.

\onslide+<4->{
For training and validation set, we sub-sampled the ``no'' examples so that the ``yes'' class and the ``no'' class both had \num{6602} examples.
}

\begin{table}
\centering
\caption{Statistics of the dataset}
\label{dataset-stats}
\setlength\tabcolsep{10pt}
\begin{tabular}{lrrr}
Name 					& \# Yes 					& \# No 					& Total 			\\\hline
\onslide<3->{Training and validation} & \onslide<3->{\num{6602}} 	& \onslide<4->{\num{6602}} 	& \onslide<4->{\num{13204}} \\
\onslide<2->{%
Test         			& \num{738}   			& \num{6469}  			& \num{7207} 	\\
}%
\onslide<3->{Unused}       			& \onslide<3->{0}	    					& \onslide<5->{\num{51668}}			& \onslide<5->{\num{51668}} 	\\\hline
Filtered dataset		& \num{7340} 			& \num{64739}			& \num{72079}	\\\hline
\end{tabular}
\end{table}

\end{frame}

\begin{frame}{Evaluation}

\begin{itemize}
\item After we found the best model from training and validation, we saved the model and ran it against the unseen test set.
\item<3-> Accuracy is 0.8565, false-positive rate is 0.0203.
\item<4-> Precision is 0.9745, recall is 0.8627.
\end{itemize}

\onslide<2->
\begin{table}
\centering
\caption{Confusion matrix for the test set}
\label{confusion-matrix}
\smallskip % add additional distance between caption and the table
\setlength\tabcolsep{10pt}
\begin{tabular}{|l|r|r|}
	\hline
	& pred. ``yes'' 	& pred. ``no'' \\ \hline
	actual ``yes'' & 5581        		& 888        \\ \hline
	actual ``no''  & 146		 		& 592       \\ \hline
\end{tabular}
\end{table}
\end{frame}


\begin{frame}{Comparison with other work}
\begin{itemize}
\item RefDiff (2.0)~\cite{silva2020refdiff} and RMiner~\cite{tsantalis2018accurate} read the complete content of the changed files (from SHA) before and after a commit and construct a diff of an internal format, from which they detect refactoring types.
\item They do not accept the git diff format which contains barely changed lines of changed files.
\item They detect more refactoring types (move class, extract methods, etc.); distinguish method renaming and class renaming, but ignore variable and field renaming.

\end{itemize}


\end{frame}

\begin{frame}{Comparison with other work}


\begin{table}[t]
\caption{Comparison of precision and recall among our neural network approach, RefDiff, and RMiner}
\label{nn-refdiff-rminer}
\renewcommand*{\arraystretch}{1.5}
\tiny
\begin{tabular}{|l|l|l|l|l|l|l|}
\hline
& \multicolumn{2}{c|}{NN}                              & \multicolumn{2}{c|}{RefDiff 2.0} & \multicolumn{2}{c|}{RMiner} \\ \hline
Refactoring Types & Precision                  & Recall                  & Precision        & Recall        & Precision      & Recall     \\ \hline
Other renaming    & \multirow{3}{*}{0.9745} & \multirow{3}{*}{0.8627} &                  &               &                &            \\ \cline{1-1} \cline{4-7}
Class renaming    &                            &                         & 0.922            & 0.874         & 0.983          & 0.621      \\ \cline{1-1} \cline{4-7}
Method renaming   &                            &                         & 0.946            & 0.694         & 0.978          & 0.771      \\ \hline
\end{tabular}
\end{table}
\end{frame}


\begin{frame}{Classification Speed}

\begin{itemize}
	\item The speed unit is example per second (eps).
	\item RefDiff 2.0 requires the SHA1 of a git commit for computing its CST diff.
	Hence we transformed our filtered dataset to this format and fed to RefDiff 2.0.
\end{itemize}


\begin{figure}
\centering
\begin{tikzpicture}[scale=0.8]
\begin{axis}[
	symbolic x coords={Antlr, NN, RefDiff 2.0},
	xtick=data,
	nodes near coords]
	\addplot[ybar,fill=blue] coordinates {
		(NN, 480.47)
		(Antlr, 3.00)
		(RefDiff 2.0, 14.26)
	};
\end{axis}
\end{tikzpicture}
\end{figure}

\end{frame}


\begin{frame}{Generalization}
\begin{itemize}
\item Generalize to other programming languages
	\begin{itemize}
	\item The text vectorization layer may need to be changed.
	\begin{itemize}
		\item Java: .123, 456\_789
		\item Python: spaces are essential.
		\item Allowed symbols in identifier: dollar sign, underscore, etc.
	\end{itemize}
	\item The word embeddings should be trained from a corpus of that language.
	\end{itemize}
\item<2-> Generalize to other refactoring types
	\begin{itemize}
		\item Keep the text vectorization layer the same
		\item Use other neural layers, e.g. LSTM, GRU
	\end{itemize}
\end{itemize}
\end{frame}

\section{Conclusion}
\begin{frame}{Conclusion}
\begin{itemize}
\item A novel neural network approach to classify code commits for identifier renaming,
only based on the changed part, without the full source code,
to assist the code review process
\item A dataset that labels \num{73080} Java diff files with whether they are renaming
\item An {\sc antlr} syntax analyzer for Java diff files
\item The accuracy of our model is 0.8565 against unseen data, false positive rate is 0.0203.
\item Our NN model runs faster comparing to traditional syntax analyses.
\end{itemize}

\end{frame}



\begin{frame}[t,allowframebreaks]{References}
	\bibliographystyle{ieeetr}
	\bibliography{bibliography/bibliography.bib,bibliography/renaming.bib}
\end{frame}

%
%{
%	\usebackgroundtemplate{\includegraphics[height=\paperheight]{fixie.jpg}}%
%	\contourlength{.06em}
%	\begin{frame}
%	\onslide<2->
%	\begin{tikzpicture}[overlay]
%	\node[yshift=1cm] at (current page.center) {\contour{white}{\Huge Thank you!}};
%	\end{tikzpicture}
%	\end{frame}
%}

\end{document}
